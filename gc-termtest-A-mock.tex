% gc-termtest-A-mock.tex

\documentclass[11pt]{article}
\usepackage{alltt}
\usepackage{enumerate}
\usepackage{syllogism} 
\usepackage{october}
\usepackage[table]{xcolor}

\begin{document}

\textbf{Term Test A Material Covered}

\begin{description}
\item[Functions] find the domain and range of a function
\item[Limits] use properties of limits to find the limit of a
  function
\item[Basic Differentiation] use basic differentiation rules to find
  derivatives
\item[Product and Quotient Rule] use product and quotient rules
  to find derivatives
\item[Trigonometric Differentiation] know the derivatives of
  $\sin{}x,\cos{}x,\tan{}x,\arcsin{}x,\arctan{}x$
\end{description}

\textbf{Term Test A Practice Questions}

(1) Find the limits for the following expressions.

\begin{equation}
  \label{eq:ageebota}
  \lim_{x\rightarrow{}0}\sqrt{x^{3}-x}
\end{equation}

\begin{equation}
  \label{eq:lietefie}
  \lim_{x\rightarrow{}-\frac{5}{2}}\frac{2x+5}{5x+2}
\end{equation}

\begin{equation}
  \label{eq:waicahng}
  \lim_{x\rightarrow{}1}\frac{2x+3}{x^{2}+x-2}
\end{equation}

\begin{equation}
  \label{eq:moluoloh}
  \lim_{x\rightarrow{}1}\frac{\sqrt{x^{2}-x}}{x-x^{2}}
\end{equation}

(2) Find the derivative of the following functions.

\begin{equation}
  \label{eq:teeheeko}
  f(x)=\frac{3}{4}\sqrt{2-x}\mbox{, using the definition of derivatives}
\end{equation}

\begin{equation}
  \label{eq:xaumaezi}
  f(x)=\frac{6}{x^{3}}+\frac{2}{x^{2}}-2
\end{equation}

\begin{equation}
  \label{eq:utazeafa}
  f(x)=\frac{3}{x+\sqrt{x}}
\end{equation}

\begin{equation}
  \label{eq:feivooch}
  f(x)=\frac{x}{2x+\frac{1}{3x+1}}
\end{equation}

\begin{equation}
  \label{eq:feeraiva}
  h(r)=\sqrt{4r+\frac{3}{r^{2}+1}}
\end{equation}

\begin{equation}
  \label{eq:siebaish}
  f(x)=\left(\sqrt{x}+\frac{1}{1+\sqrt{x^{2}+2}}\right)^{2}
\end{equation}

\begin{equation}
  \label{eq:feiveloh}
  f(\vartheta)=\sqrt{\cos{}2\vartheta}
\end{equation}

\begin{equation}
  \label{eq:oifahtie}
  G(t)=\sin\sqrt{\vert{}t\vert}
\end{equation}

\begin{equation}
  \label{eq:isaithar}
  f(x)=(1-x)^{2}\cos\frac{1}{x}
\end{equation}

\begin{equation}
  \label{eq:ixoathat}
  f(x)=\ln\ln{}x
\end{equation}

\begin{equation}
  \label{eq:airougof}
  f(x)=\frac{e^{x}-e^{-x}}{e^{x}+e^{-x}}
\end{equation}

(3) Find the equation of the tangent line for the following functions
and points.

\begin{equation}
  \label{eq:fiedaghe}
  f(x)=\frac{2}{3-4\sqrt{x}}, P=(1,?)
\end{equation}

\begin{equation}
  \label{eq:oozeewai}
  v(t)=\frac{1+\sqrt{t}}{1-\sqrt{t}}, P=(?,-3)
\end{equation}

\begin{equation}
  \label{eq:heijeivo}
  f(x)=\sqrt{1+2x^{2}}, P=(2,?)
\end{equation}

\begin{equation}
  \label{eq:jeimouti}
  f(x)=e^{2x}, P=(2,?)
\end{equation}

(4) Find the second derivatives for the following functions.

\begin{equation}
  \label{eq:eaghohpo}
  f(x)=x^{\frac{1}{3}}-x^{-\frac{1}{3}}
\end{equation}

\begin{equation}
  \label{eq:boopeesh}
  f(x)=(x^{2}+3)\sqrt{x}
\end{equation}

(5) A radioactive substance decays at a rate proportional to the
amount present. If 30 percent of such a substance decays in 15 years,
what is the half-life of the substance?

(6) If a body in a room warms up from $5^{\circ}C$ to $10^{\circ}C$ in
4 minutes, and if the room is being maintained at 20$^{\circ}C$, how
much longer will the body take to warm up to $15^{\circ}C$?

(7) Analyze the following functions. Indicate whether there are
asymptotes; calculate their line equations only if you wish.

\begin{equation}
  \label{eq:ahkeique}
  f(x)=4-\frac{e^{x}+1}{e^{x}}
\end{equation}

\begin{equation}
  \label{eq:mohsixox}
  f(x)=\frac{3x^{2}-5}{x-2}
\end{equation}

\end{document}

