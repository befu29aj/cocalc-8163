% gc-lab-14-integration.tex

\documentclass[11pt]{article}
\usepackage{enumerate}
\usepackage{syllogism} 
\usepackage{october}
\usepackage[table]{xcolor}

\newcounter{aufg}
\setcounter{aufg}{0}
\newcommand{\aufgabe}[0]{\refstepcounter{aufg}\textbf{(\arabic{aufg})}}

\begin{document}

\textbf{Integration}

{\aufgabe} Find the length of the following curve:

\begin{equation}
  \label{eq:lipafohp}
  y=\frac{1}{3}\left(x^{2}+2\right)^{\frac{3}{2}}\mbox{ from  }x=0\mbox{ to }x=3\notag
\end{equation}

{\aufgabe} Find the length of the following curve:

\begin{equation}
  \label{eq:dohpaiph}
  x=\frac{y^{3}}{3}+\frac{1}{4y}\mbox{ from }y=1\mbox{ to }y=3\notag
\end{equation}

{\aufgabe} Find the length of the following curve:

\begin{equation}
  \label{eq:aijaishu}
  y=\frac{x^{2}}{2}-\frac{\ln{}x}{4}\mbox{ from }x=1\mbox{ to }x=3\notag
\end{equation}

{\aufgabe} Find the area of a surface of revolution for the following curve:

\begin{equation}
  \label{eq:ohngeigi}
  y=2\sqrt{x},1\leq{}x\leq{}2\notag
\end{equation}

{\aufgabe} Evaluate the following indefinite integral:

\begin{equation}
  \label{eq:reivoumo}
  \int_{1}^{\infty}\frac{\ln{}x}{x^{2}}\,dx\notag
\end{equation}

{\aufgabe} Evaluate the following indefinite integral:

\begin{equation}
  \label{eq:paisiefu}
  \int_{0}^{\infty}e^{-\frac{x}{2}}\,dx\notag
\end{equation}

{\aufgabe} Use integration by parts to find the following integral:

\begin{equation}
  \label{eq:nophieyu}
  \int\left(\ln{}x\right)^{2}\,dx\notag
\end{equation}

{\aufgabe} Use integration by parts to find the following integral:

\begin{equation}
  \label{eq:aejuvahg}
  \int_{0}^{0.5\pi^{2}}\cos\sqrt{2x}\,dx\notag
\end{equation}

{\aufgabe} Use integration by parts or trigonometric integration to find the following integral:

\begin{equation}
  \label{eq:kiedalei}
  \int_{\frac{\pi}{2}}^{\pi}\sin^{4}x\,dx\notag
\end{equation}

{\aufgabe} Evaluate the following definite integral:

\begin{equation}
  \label{eq:pahteeth}
\int_{1}^{3}x\sqrt{3x^{2}-2}\,dx\notag
\end{equation}

{\aufgabe} Evaluate the following definite integral:

\begin{equation}
  \label{eq:riweevie}
\int_{0}^{2}\frac{x}{\sqrt{x^{2}+5}}\,dx\notag
\end{equation}

{\aufgabe} Evaluate the following definite integral:

\begin{equation}
  \label{eq:omixughu}
  \int_{0}^{2}xe^{x^{2}}\,dx\notag
\end{equation}

{\aufgabe} Evaluate the following definite integral:

\begin{equation}
  \label{eq:aeteepah}
\int_{1}^{2}\frac{\ln{}x}{x}\,dx\notag\notag
\end{equation}

{\aufgabe} Evaluate the following integral.
  \begin{equation}
    \label{eq:iepahgai}
    \int_{0}^{\frac{\pi}{6}}3\cos^{5}3x\,dx\notag
  \end{equation}

\end{document}

