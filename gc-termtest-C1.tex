% gc-termtest-C.tex

\documentclass[11pt]{article}
\usepackage{alltt}
\usepackage{enumerate}
\usepackage{syllogism} 
\usepackage{october}
\usepackage[table]{xcolor}
\pagestyle{empty}

\newcounter{aufg}
\setcounter{aufg}{0}
\newcommand{\aufgabe}[1]{\refstepcounter{aufg}\textbf{(\arabic{aufg})}[#1 points]}

\begin{document}

\textbf{Term Test C version 1}

\aufgabe{5} The two curves $y=x$ and $y=x^{3}$ meet three times; call the three points of intersection $A,B,$ and $C$, from left to right. Find the area between the two curves between $A$ and $C$. If part of this area is below the $x$-axis, make sure to \emph{add} it to the total area and not \emph{subtract} it.

\aufgabe{5} Breathing is cyclic and a full respiratory cycle from the beginning of inhalation to the end of exhalation takes about 5 seconds. The maximum rate of air flow into the lungs is about 0.5 litres per second. This explains, in part, why the function
\begin{equation}f(t)=\frac{1}{2}\sin\left(\frac{2\pi}{5}t\right)
nd{equation}
has often been used to model the rate of air flow into the lungs. Use this model to find the volume of inhaled air in the lungs at time $t$.

\aufgabe{5} Find the following arc length.
\begin{equation}
\label{eq:eedookuo}
y=\frac{1}{8}x^{4}+\frac{1}{4}x^{-2},1\leq{}x\leq{}2
\end{equation}

\aufgabe{5} Use substitution to evaluate the definite integral\begin{equation}\int_{4}^{1}\frac{(\sqrt{x}+1)^{4}}{2\sqrt{x}}\,dx\end{equation}

\aufgabe{5} Find the length of the following curve.
\begin{equation}
y=\int_{0}^{x}\sqrt{\sec^{4}t-1}\,dt,-\frac{\pi}3\leq{}x\leq\frac{\pi}3
\end{equation}
Remember that according to the Fundamental Theorem of Calculus, if $g(x)=\int_{a}^{x}f(t)\,dt$, then $g'(x)=f(x)$.

\aufgabe{5} $S$ is a solid generated by revolving a bounded region $R$ about the $x$-axis. Find the volume of $S$. $R$ is bounded by the lines $y=0$, $x=\pi/6$, $x=\pi/4$, and the curve $y=\cos{}x$. You may want to use integral 66 from Thomas' Brief Table of of Integrals,
\begin{equation}
\int\cos^{2}ax\,dx=\frac{x}{2}+\frac{\sin{}2ax}{4a}+C\end{equation}

\aufgabe{5} Find the area of the surface generated by revolving about the $y$-axis the arc $C$ given by
\begin{equation}
\label{eq:aiquuojo}
x=2\sqrt{\frac{y}{3}},1\leq{}y\leq{}2
\end{equation}

\end{document}
