% gc-finalexam-2018.tex

\documentclass[11pt]{article}
\usepackage{alltt}
\usepackage{enumerate}
\usepackage{syllogism} 
\usepackage{october}
\usepackage[table]{xcolor}
\pagestyle{empty}

\newcounter{aufg}
\setcounter{aufg}{0}
\newcommand{\aufgabe}[1]{\refstepcounter{aufg}\textbf{(\arabic{aufg})}
[#1 points]}

\begin{document}

\textbf{Final Exam}

There are eight questions with a total of 60 points. You may need the
following:
% You may need one of the following trigonometric identities:
\begin{equation}
  \label{eq:eechiwoo}
  \sin(\alpha+\beta)=\sin\alpha\cos\beta+\sin\beta\cos\alpha
\end{equation}
\begin{equation}
  \label{eq:bahwapoh}
  \cos(\alpha+\beta)=\cos\alpha\cos\beta+\sin\beta\sin\alpha
\end{equation}
% \begin{equation}
%   \label{eq:youyaida}
%   \sec^{2}\vartheta-1=\tan^{2}\vartheta
% \end{equation}
\begin{equation}
  \label{eq:veiyoeno}
  \frac{d}{d\vartheta}\tan{}\vartheta=\sec^{2}\vartheta
\end{equation}

\aufgabe{4} Find an equation of the line tangent to the following
curve at the given point.
\begin{equation}
  \label{eq:cahbiech}
  y=\frac{4x}{x^{2}+3},x=3
\end{equation}
% \begin{equation}
%   \label{eq:faipheey}
%   y=3x^{3}+\sin{}x,x=\frac{\pi}{3}
% \end{equation}
% Briggs Cochran 221:41--42

\aufgabe{6} Evaluate the following limits. 
\begin{equation}
  \label{eq:yahsubee}
\lim_{x\rightarrow{}0}\frac{x-\sin{}x}{x^{3}}
\end{equation}
% Cooley 467-6
\begin{equation}
  \label{eq:zovainga}
\lim_{n\rightarrow\infty}\sum_{k=1}^{n}\frac{2^{k}}{3^{k+2}}
\end{equation}
% Briggs Cochran 587:19

\aufgabe{6} All boxes with a square base and a volume of 50ft$^{3}$
have a surface area given by
\begin{equation}
  \label{eq:aepahgua}
  S(x)=2x^{2}+\frac{200}{x}
\end{equation}
where $x$ is the length of the sides of the base. Find the absolute
minimum of the surface area function. What are the dimensions of the
box with minimum surface area?
% Briggs Cochran 231:68

\aufgabe{6} Consider the two curves $y=\sin{}x$ and $y=\sin{}(2x)$.
They intersect three times between $x=0$ and $x=\pi$, call these
intersection points $A,B,C$ from left to right. What is the total area
between these two curves from $x=B$ to $x=C$? If part of this area is
below the $x$-axis, it is added to the total area, not subtracted. 
% Briggs Cochran 388:16

\newpage

\aufgabe{12} Evaluate the following definite integrals. For
(\ref{eq:thoonohj}), present your answer as $\ln(n/m)$, where $n$ and
$m$ are whole numbers.
\begin{equation}
  \label{eq:thoonohj}
  \int_{-1}^{1}\frac{3x}{x^{2}+2x-8}\,dx
\end{equation}
% Briggs Cochran 477 and self
% \begin{equation}
%   \label{eq:veizaiva}
%   \int_{0}^{\frac{\pi}{2}}\sin^{3}(2\vartheta)\,d\vartheta
% \end{equation}
% concocted by self
\begin{equation}
  \label{eq:therudei}
  \int_{0}^{\infty}e^{-3x}\,dx
\end{equation}
% Briggs Cochran 503:1

\aufgabe{10} Consider the following three points:
\begin{equation}
  \label{eq:seceeghi}
  P=(3,1,2)\hspace{.5in}Q=(0,-6,3)\hspace{.5in}R=(-1,-5,5)
\end{equation}
$P,Q,R$ form a triangle, and
\begin{equation}
  \label{eq:aibebael}
  \vec{PQ}\times\vec{PR}=\left(
    \begin{array}{c}
      -3     & 
           1 & 
               -2 
    \end{array}
\right)
\end{equation}

\begin{enumerate}
\item Provide the line equation for the plane containing the triangle.
\item Calculate the three interior angles of the triangle.
\end{enumerate}
% concocted by self

\aufgabe{8} Find the critical points of the following function. Use
the Second Derivative Test to determine (if possible) whether each
critical point corresponds to a local maximum, local minimum, or
saddle point.
\begin{equation}
  \label{eq:sheoceif}
f(x,y)=(4x-1)^{2}+(2y+4)^{2}+1  
\end{equation}
% Briggs Cochran 861:16

\aufgabe{6} Let
\begin{equation}
  \label{eq:yeitiebe}
  f(x)=\frac{1}{1+2x}
\end{equation}
Use the Maclaurin series expansion to approximate
\begin{equation}
  \label{eq:mafohchu}
  f(x)\approx{}c_{0}+c_{1}x+c_{2}x^{2}+c_{3}x^{3}+c_{4}x^{4}
\end{equation}
Then, provide the full expansion
\begin{equation}
  \label{eq:quaizobu}
  f(x)=\sum_{n=0}^{\infty}c_{n}x^{n}
\end{equation}
% Briggs Cochran 621:14

\end{document}

\begin{equation}
  \label{eq:ieneizai}
  \begin{array}{rcl}
    \sum{}xy&=&62034 \\
    \sum{}x&=&791 \\
    \sum{}y&=&775 \\
    \sum{}x^{2}&=&63701 \\
    \sum{}y^{2}&=&60937 \\
  \end{array}
\end{equation}