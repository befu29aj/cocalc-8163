% gc-termtest-A-davidworthen.tex

\documentclass[11pt]{article}
\usepackage{alltt}
\usepackage{enumerate}
\usepackage{syllogism} 
\usepackage{october}
\usepackage[table]{xcolor}
\pagestyle{empty}

\newif\ifOneOrTwo
\OneOrTwotrue
%\OneOrTwofalse
\ifOneOrTwo
\newcommand{\eibe}{1}
\newcommand{\ufoj}{\lim_{x\rightarrow{}5}\left(\frac{1}{x-5}-\frac{10}{x^{2}-25}\right)}
\newcommand{\mair}{y=\frac{1}{5\sin{}x+2\cos{}x}}
\newcommand{\utit}{0.5}
\newcommand{\jief}{3}
\newcommand{\caib}{g(t)=-7e^{t\cos{}t}}
\newcommand{\eizi}{Polonium-210 The half-life of polonium is 139 days, but your sample will not be useful to you after 95\% of the radioactive nuclei present on the day the sample arrives has disintegrated. For about how many days after the sample arrives will you be able to use the polonium?}
\newcommand{\afie}{65}
\newcommand{\uphe}{35}
\newcommand{\kieg}{50}
% \newcommand{\aequ}{}
% \newcommand{\aiza}{}
% \newcommand{\ajoo}{}
% \newcommand{\ajoo}{}
% \newcommand{\icoo}{}
% \newcommand{\phae}{}
% \newcommand{\usha}{}
% \newcommand{\oaga}{}
% \newcommand{\cait}{}
% \newcommand{\viob}{}
% \newcommand{\mare}{}
% \newcommand{\ieje}{}
% \newcommand{\opie}{}
% \newcommand{\geik}{}
% \newcommand{\ahna}{}
% \newcommand{\xohv}{}
% \newcommand{\dahj}{}
% \newcommand{\eexe}{}
% \newcommand{\etah}{}
% \newcommand{\ieja}{}
% \newcommand{\vail}{}
% \newcommand{\sien}{}
% \newcommand{\ieti}{}
% \newcommand{\ooga}{}
% \newcommand{\oopu}{}

\else
\newcommand{\eibe}{2}
\newcommand{\ufoj}{\lim_{x\rightarrow{}4}\left(\frac{1}{x-4}-\frac{8}{x^{2}-16}\right)}
\newcommand{\mair}{y=\frac{1}{5\cos{}x+2\sin{}x}}
\newcommand{\utit}{\frac{1}{2}}
\newcommand{\jief}{2}
\newcommand{\caib}{g(t)=-7e^{t\sin{}t}}
\newcommand{\eizi}{The half-life of Palladium-100 is 4 days. After 20 days a sample of Palladium-100 has been reduced to a mass of 1 mg. What was the initial mass (in mg) of the sample?}
\newcommand{\afie}{70}
\newcommand{\uphe}{30}
\newcommand{\kieg}{45}
% \newcommand{\aequ}{}
% \newcommand{\aiza}{}
% \newcommand{\ajoo}{}
% \newcommand{\ajoo}{}
% \newcommand{\icoo}{}
% \newcommand{\phae}{}
% \newcommand{\usha}{}
% \newcommand{\oaga}{}
% \newcommand{\cait}{}
% \newcommand{\viob}{}
% \newcommand{\mare}{}
% \newcommand{\ieje}{}
% \newcommand{\opie}{}
% \newcommand{\geik}{}
% \newcommand{\ahna}{}
% \newcommand{\xohv}{}
% \newcommand{\dahj}{}
% \newcommand{\eexe}{}
% \newcommand{\etah}{}
% \newcommand{\ieja}{}
% \newcommand{\vail}{}
% \newcommand{\sien}{}
% \newcommand{\ieti}{}
% \newcommand{\ooga}{}
% \newcommand{\oopu}{}
\fi

\newcounter{aufg}
\setcounter{aufg}{0}
\newcommand{\aufgabe}[1]{\refstepcounter{aufg}\textbf{(\arabic{aufg})}
[#1 points]}

\begin{document}

\textbf{Term Test A}

\aufgabe{5} Evaluate the limit
\begin{equation}
  \label{eq:waengohn}
  \lim_{x\rightarrow{}-1}\frac{2x^{2}-x-3}{x+1}\notag
\end{equation}

\aufgabe{8} Find the equation of the tangent line to the curve
\begin{equation}
  \label{eq:leegaigh}
  y=\sqrt{x^{2}+1}\cdot\tan\left(\frac{x}{2}\right)\notag
\end{equation}
at the point $(0,0)$. 

\aufgabe{7} Find the derivative of
\begin{equation}
  \label{eq:iefaigho}
  f(x)=\arctan\left(x-\sqrt{1+x^{2}}\right)
\end{equation}

\aufgabe{5} Find the domain and the derivative of 
\begin{equation}
  \label{eq:ohzabeed}
g(z)=\frac{e^{z}-1}{2+\ln{}z}  
\end{equation}

\aufgabe{7} The analysis of tooth shrinkage by C. Loring Brace and
colleagues at the University of Michigan's Museum of Anthropology
indicates that human tooth size is continuing to decrease and that the
evolutionary process did not come to a halt some 30,000 years ago as
many scientists contend. In northern Europeans, for example, tooth
size reduction now has a rate of 1\% per 1,000 years. In about how
many years will human teeth be 90\% of their present size?

\aufgabe{8} Provide the second derivative of the following two
functions in their simplest form. 
\begin{equation}
  \label{eq:loesheop}
  f(x)=\frac{x^{2}}{2}\left(\ln\frac{x}{2}\right)-x
\end{equation}
\begin{equation}
  \label{eq:eighohbi}
  g(\vartheta)=\vartheta\sin\vartheta\notag
\end{equation}

\aufgabe{9} The temperature of an ingot of silver is 60$^{\circ}$C above room
  temperature right now. Twenty minutes ago, it was 70$^{\circ}$C above room
  temperature. How far above room temperature will the silver be
  \begin{enumerate}
  \item Fifteen minutes from now? 
  \item Two hours from now? 
  \item When will the silver be 10$^{\circ}$C above room temperature? 
  \end{enumerate}

\end{document}








