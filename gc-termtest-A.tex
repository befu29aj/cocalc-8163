%%%%%%%%%%%%%%%%%%%%%%%%%%%%%%%%%%%%%%%%%%%%%%%%%%%%%%%%%%%%%%%%%%
%%%%%%%        NEW PROCEDURE: use gc-termtest-A.pl         %%%%%%%
%%%%%%%%%%%%%%%%%%%%%%%%%%%%%%%%%%%%%%%%%%%%%%%%%%%%%%%%%%%%%%%%%%

% gc-termtest-A.tex

\documentclass[11pt]{article}
\usepackage{alltt}
\usepackage{enumerate}
\usepackage{syllogism} 
\usepackage{october}
\usepackage[table]{xcolor}
\pagestyle{empty}

\newif\ifOneOrTwo
\OneOrTwotrue
%\OneOrTwofalse
\ifOneOrTwo
\newcommand{\eibe}{1}
\newcommand{\ufoj}{\lim_{x\rightarrow{}5}\left(\frac{1}{x-5}-\frac{10}{x^{2}-25}\right)}
\newcommand{\mair}{y=\frac{1}{5\sin{}x+2\cos{}x}}
\newcommand{\utit}{0.5}
\newcommand{\jief}{3}
\newcommand{\caib}{g(t)=-7e^{t\cos{}t}}
\newcommand{\eizi}{Polonium-210 The half-life of polonium is 139 days, but your sample will not be useful to you after 95\% of the radioactive nuclei present on the day the sample arrives has disintegrated. For about how many days after the sample arrives will you be able to use the polonium?}
\newcommand{\afie}{65}
\newcommand{\uphe}{35}
\newcommand{\kieg}{50}
% \newcommand{\aequ}{}
% \newcommand{\aiza}{}
% \newcommand{\ajoo}{}
% \newcommand{\ajoo}{}
% \newcommand{\icoo}{}
% \newcommand{\phae}{}
% \newcommand{\usha}{}
% \newcommand{\oaga}{}
% \newcommand{\cait}{}
% \newcommand{\viob}{}
% \newcommand{\mare}{}
% \newcommand{\ieje}{}
% \newcommand{\opie}{}
% \newcommand{\geik}{}
% \newcommand{\ahna}{}
% \newcommand{\xohv}{}
% \newcommand{\dahj}{}
% \newcommand{\eexe}{}
% \newcommand{\etah}{}
% \newcommand{\ieja}{}
% \newcommand{\vail}{}
% \newcommand{\sien}{}
% \newcommand{\ieti}{}
% \newcommand{\ooga}{}
% \newcommand{\oopu}{}

\else
\newcommand{\eibe}{2}
\newcommand{\ufoj}{\lim_{x\rightarrow{}4}\left(\frac{1}{x-4}-\frac{8}{x^{2}-16}\right)}
\newcommand{\mair}{y=\frac{1}{5\cos{}x+2\sin{}x}}
\newcommand{\utit}{\frac{1}{2}}
\newcommand{\jief}{2}
\newcommand{\caib}{g(t)=-7e^{t\sin{}t}}
\newcommand{\eizi}{The half-life of Palladium-100 is 4 days. After 20 days a sample of Palladium-100 has been reduced to a mass of 1 mg. What was the initial mass (in mg) of the sample?}
\newcommand{\afie}{70}
\newcommand{\uphe}{30}
\newcommand{\kieg}{45}
% \newcommand{\aequ}{}
% \newcommand{\aiza}{}
% \newcommand{\ajoo}{}
% \newcommand{\ajoo}{}
% \newcommand{\icoo}{}
% \newcommand{\phae}{}
% \newcommand{\usha}{}
% \newcommand{\oaga}{}
% \newcommand{\cait}{}
% \newcommand{\viob}{}
% \newcommand{\mare}{}
% \newcommand{\ieje}{}
% \newcommand{\opie}{}
% \newcommand{\geik}{}
% \newcommand{\ahna}{}
% \newcommand{\xohv}{}
% \newcommand{\dahj}{}
% \newcommand{\eexe}{}
% \newcommand{\etah}{}
% \newcommand{\ieja}{}
% \newcommand{\vail}{}
% \newcommand{\sien}{}
% \newcommand{\ieti}{}
% \newcommand{\ooga}{}
% \newcommand{\oopu}{}
\fi

\newcounter{aufg}
\setcounter{aufg}{0}
\newcommand{\aufgabe}[1]{\refstepcounter{aufg}\textbf{(\arabic{aufg})}
[#1 points]}

\begin{document}

\textbf{Term Test A version {\eibe}}

\aufgabe{5} Evaluate the limit
\begin{equation}
  \label{eq:wohghoda}
  {\ufoj}\notag
\end{equation}

\aufgabe{8} Find the equation of the tangent line to the curve
\begin{equation}
  \label{eq:leegaigh}
  {\mair}\notag
\end{equation}
at the point $(0,{\utit})$. 

\aufgabe{7} Find the domain and the derivative of
\begin{equation}
  \label{eq:javahvap}
  f(x)=\frac{\ln{}x}{{\jief}+\ln{}x}\notag
\end{equation}
Make sure to simplify the derivative as much as possible.

\aufgabe{5} Find the derivative of 
\begin{equation}
  \label{eq:ohjoceej}
  {\caib}\notag
\end{equation}

\aufgabe{7} {\eizi}

\aufgabe{8} Provide the second derivative of the following two
functions in their simplest form. 
\begin{equation}
  \label{eq:loesheop}
  % f(x)=\frac{2x^{2}+7x+3}{x-1}\notag
  f(x)=5x-\frac{3}{x-2}\notag
\end{equation}
\begin{equation}
  \label{eq:eighohbi}
  g(x)=\frac{x^{2}\ln{}x}{2}-\frac{3x^{2}}{4}\notag
\end{equation}

\aufgabe{9} An aluminum beam was brought from the outside cold
into a machine shop where the temperature was held at
{\afie}$^{\circ}$F. After 10 minutes, the beam warmed to
{\uphe}$^{\circ}$F and after another 10 minutes it was
{\kieg}$^{\circ}$F. Use Newton's law of cooling to estimate the
beam's initial temperature.

\end{document}








