% gc-termtest-A.tex

\documentclass[11pt]{article}
\usepackage{alltt}
\usepackage{enumerate}
\usepackage{syllogism} 
\usepackage{october}
\usepackage[table]{xcolor}
\pagestyle{empty}

% testing overleaf

\newcounter{aufg}
\setcounter{aufg}{0}
\newcommand{\aufgabe}[1]{\refstepcounter{aufg}\textbf{(\arabic{aufg})}
[#1 points]}

\begin{document}

\textbf{Term Test A}

\aufgabe{5} Evaluate the limit
\begin{equation}
  \label{eq:wohghoda}
  \lim_{x\rightarrow{}5}\left(\frac{1}{x-5}-\frac{10}{x^{2}-25}\right)\notag
\end{equation}

\aufgabe{8} Find the equation of the tangent line to the curve
\begin{equation}
  \label{eq:leegaigh}
  y=\frac{1}{5\sin{}x+2\cos{}x}\notag
\end{equation}
at the point $(0,0.5)$. 

\aufgabe{7} Find the domain and the derivative of
\begin{equation}
  \label{eq:javahvap}
  f(x)=\frac{\ln{}x}{3+\ln{}x}\notag
\end{equation}
Make sure to simplify the derivative as much as possible.

\aufgabe{5} Find the derivative of 
\begin{equation}
  \label{eq:ohjoceej}
  g(t)=-7e^{t\sin{}t}\notag
\end{equation}

\aufgabe{7} The half-life of Palladium-100 is 4 days. After 20 days a
sample of Palladium-100 has been reduced to a mass of 1 mg. What was
the initial mass (in mg) of the sample?

\aufgabe{8} Find the $x$-intercepts, critical points, and inflection
points of 
\begin{equation}
  \label{eq:loesheop}
  f(x)=\frac{2x^{2}+7x+3}{x-1}\notag
\end{equation}
Provide the second derivative in its simplest form. If there are
asymptotes for the function $f(x)$, specify their slope $k$
($k=\infty$ for a vertical asymptote, $k=0$ for a horizontal
asymptote, and $k\neq{}0$ for a slanted asymptote).

Just saying, 2017-12-08, that this is a beautiful function to analyze:

\begin{equation}
  \label{eq:phauruwi}
  f(x)=\frac{x-1}{3x^{2}}
\end{equation}

the calculations are in ft-lab-20171208.pdf in graffiti.

% \begin{equation}
%   \label{eq:pomaifah}
%   -\frac{2 \, {\left(4 \, x + 7\right)}}{{\left(x - 1\right)}^{2}} + \frac{4}{x - 1} + \frac{2 \, {\left(2 \, x^{2} + 7 \, x + 3\right)}}{{\left(x - 1\right)}^{3}}\notag
% \end{equation}

\end{document}

