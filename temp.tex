\aufgabe{5} The two curves $y=x$ and $y=x^{3}$ meet three times; call the three points of intersection $A,B,$ and $C$, from left to right. Find the area between the two curves between $A$ and $C$. If part of this area is below the $x$-axis, make sure to \emph{add} it to the total area and not \emph{subtract} it.
% Protter and Protter, page 224, example 2

\aufgabe{5} $S$ is a solid generated by revolving a bounded region $R$ about the $x$-axis. Find the volume of $S$. $R$ is bounded by the {\ufoj}. {\afie}
% Protter and Protter, page 232, exercise 27 and 28

\aufgabe{5} Find the area of the surface generated by revolving about the $y$-axis the arc $C$ given by
\begin{equation}
  \label{eq:aiquuojo}
  x={\uphe}\sqrt{\frac{y}{\kieg}},1\leq{}y\leq{}2  
\end{equation}
% Protter and Protter, page 245, example 3

\aufgabe{5} Find the following arc length.
\begin{equation}
  \label{eq:eedookuo}
  {\mair}
\end{equation}
% Protter and Protter, page 241, exercises 9 and 11

\aufgabe{5} Evaluate the following integral.
\begin{equation}
  \label{eq:teingahg}
  \int_{{\caib}}^{{\eizi}}{\utit}^{3}{\jief}t\,dt
\end{equation}

\aufgabe{5} Find the length of the following curve.

\begin{equation}
  \label{eq:thoageib}
  x=\int_{0}^{y}\sqrt{\sec^{4}t-1}\,dt,-\frac{\pi}{\aequ}\leq{}y\leq\frac{\pi}{\aequ}
\end{equation}

\aufgabe{5} Use integration by parts to find the following integral. Remember that you can find the anti-derivative of $f(x)=\ln{}x$ by writing $f(x)=\ln{}x\cdot{}1$ and then integrating by parts.

\begin{equation}
  \label{eq:nophieyu}
  \int\left(\ln{}x\right)^{2}\,dx
\end{equation}
